\documentclass[12pt,a4paper]{article}
\usepackage[utf8]{inputenc}
\usepackage[english,russian]{babel}
\usepackage{hyperref}
\hypersetup{
	colorlinks   = true, 
	urlcolor     = blue, 
	linkcolor    = blue, 
	citecolor   = red
}
\usepackage{amsthm,amssymb,amsfonts,amsmath}
\usepackage{physics}
\usepackage{xcolor}
\usepackage[left=3cm,right=3cm,
top=3cm,bottom=3cm,bindingoffset=0cm]{geometry}
\usepackage{tcolorbox}

\newtcolorbox{exercise}[1][]
{
	colframe = black!20,
	colback  = white,
	coltitle = black,  
	title    = {#1},
}


\author{Самутичев Е.Р.}
\date{\today}
\title{}


\begin{document}
	
\maketitle

\begin{exercise}[Задача 1]
	Имеется цепь Маркова с матрицей перехода:
	$$\mathcal{P} = \begin{pmatrix}
			0 & \frac{1}{2} & \frac{1}{2} \\[4pt]
			\frac{1}{3} & \frac{1}{3} & \frac{1}{3} \\[4pt]
			\frac{1}{4} & \frac{2}{4} & \frac{1}{4}
		\end{pmatrix}$$
	Найти вероятность $p_{3,2}^{(3)}$.
\end{exercise}
\begin{proof}[Решение]
	Известно что матрица перехода за $n$ шагов есть $n$-ая степень матрицы перехода за один шаг, воспользовавшись этим получаем:
	$$p_{3, 2}^{(3)} = (\mathcal{P}^3)_{3, 2}$$
	Для удобства приведем строки к общему знаменателю и вынесем за скобку:
	$$\mathcal{P} = \frac{1}{12} \begin{pmatrix}
		0 & 6 & 6 \\
		4 & 4 & 4 \\
		3 & 6 & 3
	\end{pmatrix}$$
	Тогда:
	$$\mathcal{P}^2 = \frac{1}{12^2} \begin{pmatrix}
		42 & 60 & 42 \\
		28 & 64 & 52 \\
		33 & 60 & 51
	\end{pmatrix}$$
	$$\mathcal{P}^3 = \frac{1}{12^3} \begin{pmatrix}
		366 & 744 & 618 \\
		412 & 736 & 580 \\
		393 & 744 & 591
	\end{pmatrix}$$
	Итого $p_{3, 2}^{(3)} = \frac{744}{1728} = \frac{93}{216} = \frac{31}{72}$.
\end{proof}


\end{document}
