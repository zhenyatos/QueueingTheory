\documentclass[12pt,a4paper]{article}
\usepackage[utf8]{inputenc}
\usepackage[english,russian]{babel}
\usepackage{hyperref}
\hypersetup{
	colorlinks   = true, 
	urlcolor     = blue, 
	linkcolor    = blue, 
	citecolor   = red
}
\usepackage{amsthm,amssymb,amsfonts,amsmath}
\usepackage{physics}
\usepackage{xcolor}
\usepackage[left=3cm,right=3cm,
top=3cm,bottom=3cm,bindingoffset=0cm]{geometry}
\usepackage{tcolorbox}
\usepackage{import}
\usepackage{xifthen}

\newtcolorbox{exercise}[1][]
{
	colframe = black!20,
	colback  = white,
	coltitle = black,  
	title    = {#1},
}

\newcommand{\incfig}[1]{%
	\def\svgwidth{\columnwidth}
	\import{./figures/}{#1.pdf_tex}
}
\newcommand{\pP}{\mathbf{P}}
\newcommand{\mean}[1]{\mathbf{M}[#1]}

\author{Самутичев Е.Р.}
\date{\today}
\title{}



\begin{document}
	
\maketitle

\begin{exercise}[Задача 8]
	Рассмотрим систему с ожиданием на вход которой поступает простейший поток требований с интенсивностью $\lambda$. Единственный канал обслуживания обслуживает \textbf{пары} требований по показательному закону с интенсивностью $\mu$. Если в очереди остается только \textbf{одно} требование, то оно также поступает на обслуживание, интенсивность обслуживания при этом также равняется $\mu$. Найти среднее число требований находящихся в системе в установившемся режиме. 
\end{exercise}
\begin{proof}[Решение]
	Изобразим размеченный граф состояний системы. Пара $(n, s)$ означает что в накопителе ожидают $n \in \mathbb{N}\cup{0}$ требований, а обслуживаются $s = 1, 2$ требований. Таким образом, в системе находится $n + s$ требований. {\color{red}Красными} стрелками обозначим переходы с интенсивностью {\color{red}$\lambda$} соответствующие поступлению очередного требования в систему, а {\color{blue}синими} стрелками обозначим переходы с интенсивностью {\color{blue}$\mu$} соответствующие завершению обслуживания в канале.
	\begin{figure}[h!]
		\centering
		\scalebox{0.8}{\incfig{4}}
	\end{figure}

	При анализе системы будем считать что
	\begin{equation*}
		0 < \lambda < 2\mu
	\end{equation*}
	т.к. максимальное число требований на обслуживании равняется двум и вместе они обслуживаются с интенсивностью $\mu$, значит если интенсивность $\lambda$ потока требований превысит $2\mu$, получаем хаотические скопления требований и анализ такой системы затруднен. Понятно что в случае $\lambda = 0$ -- система тривиальна, требования попросту не поступают.
	
	Запишем уравнения Колмогорова для финальных вероятностей, вначале для частных случаев (состояния $(0, 0), \ (0, 1), \ (0, 2), \ (1, 1), \ (1, 2)$):
	\begin{align}\label{eq:1}
	\begin{split}
		&\mu P_{(0, 1)} + \mu P_{(0, 2)} - \lambda P_{(0, 0)} = 0 \\
		&\lambda P_{(0, 0)} + \mu P_{(1, 1)} + \mu P_{(1, 2)} - (\mu + \lambda) P_{(0, 1)} = 0 \\
		&\mu P_{(2, 1)} + \mu P_{(2, 2)} - (\mu + \lambda) P_{(0, 2)} = 0 \\
		&\lambda P_{(0, 1)} - (\mu + \lambda) P_{(1, 1)} = 0 \\
		&\lambda P_{(0, 2)} + \mu P_{(3, 1)} + \mu P_{(3, 2)} - (\mu + \lambda) P_{(1, 2)} = 0 
	\end{split}
	\end{align}
	
	Для оставшихся состояний $P_{(n, 1)}, \ P_{(n, 2)}$ при $n \geq 2$:
	\begin{align}\label{eq:2}
	\begin{split}
		&\lambda P_{(n-1, 1)} - (\mu + \lambda) P_{(n, 1)} = 0 \\
		&\lambda P_{(n-1, 2)} + \mu P_{(n+2, 1)} + \mu P_{(n+2, 2)} - (\mu + \lambda) P_{(n, 2)} = 0
	\end{split}
	\end{align}
	
	Добавим условие нормировки:
	\begin{equation}
		P_{(0, 0)} + P_{(0, 1)} + P_{(0, 2)} + \sum_{n=1}^{\infty}{P_{(n, 1)}} + \sum_{n=1}^{\infty}{P_{(n, 2)}} = 1
	\end{equation}
	
	Используя 4-ое уравнение из \eqref{eq:1} и 1-ое из \eqref{eq:2} получаем для $n \geq 0$:
	\begin{equation}\label{eq:3}
		P_{(n, 1)} = \left(\frac{\lambda}{\mu + \lambda}\right)^n P_{(0, 1)}
	\end{equation}
	
	Используя первые 2 уравнения из \eqref{eq:1} вместе с \eqref{eq:3} получаем:
	\begin{align}\label{eq:4}
	\begin{split}
		&P_{(0, 2)} = \frac{\lambda}{\mu} P_{(0, 0)} - P_{(0, 1)} \\
		&P_{(1, 2)} = \frac{\mu + \lambda}{\mu} P_{(0, 1)} - \frac{\lambda}{\mu + \lambda} P_{(0, 1)} - \frac{\lambda}{\mu} P_{(0, 0)}
	\end{split}
	\end{align}
	
	Воспользуемся методом производящих функций, определим
	\begin{equation}\label{eq:5}
		G(z) = P_{(0, 0)} + P_{(0, 1)} z + P_{(0, 2)} z^2 + \sum_{n=1}^{\infty}{P_{(n, 1)} z^{n+1}} + \sum_{n=1}^{\infty}{P_{(n, 2)} z^{n+2}}
	\end{equation}
	
	Введем обозначение:
	\begin{equation}\label{eq:a}
		a = \frac{\lambda}{\mu + \lambda}
	\end{equation}
	
	Используя \eqref{eq:3} преобразуем \eqref{eq:5}:
	\begin{multline}\label{eq:7}
		G(z) = P_{(0, 0)} + P_{(0, 1)} z + P_{(0, 2)} z^2 + P_{(0, 1)} z \sum_{n=1}^{\infty}{a^n z^n} + \sum_{n=1}^{\infty}{P_{(n, 2)} z^{n+2}} = \\ 
		= P_{(0, 0)} + P_{(0, 1)} z + P_{(0, 2)} z^2 + P_{(0, 1)} \frac{az}{1 - az} + \sum_{n=1}^{\infty}{P_{(n, 2)} z^{n+2}}
	\end{multline}
	
	Используя \eqref{eq:4} преобразуем \eqref{eq:7}:
	\begin{equation}\label{eq:8}
		G(z) = P_{(0, 0)} + P_{(0, 1)} z + \left(\frac{\lambda}{\mu} P_{(0, 0)} - P_{(0, 1)}\right) z^2 + P_{(0, 1)} \frac{az}{1 - az} + \sum_{n=1}^{\infty}{P_{(n, 2)} z^{n+2}}
	\end{equation}
	
	Отдельно рассмотрим
	\begin{equation*}
		R(z) = \sum_{n=1}^{\infty}{P_{(n, 2)} z^{n+2}}
	\end{equation*}
	и заметим что в силу условия нормировки:
	\begin{equation}\label{eq:11}
		R(1) = \sum_{n=1}^{\infty}{P_{(n, 2)}} \leq 1 < \infty
	\end{equation}
	так что
	\begin{equation}\label{eq:9}
		R(z) = \sum_{n=1}^{\infty}{P_{(n, 2)} z^{n+2}} < \infty, \forall z \in (-1, 1)
	\end{equation}
	
	Воспользуемся 5-м уравнением из \eqref{eq:1} и 2-м из \eqref{eq:2}:
	\begin{multline*}\label{eq:6}
		R(z) = \frac{1}{\mu + \lambda} \sum_{n=1}^{\infty}{\left(\lambda P_{(n-1, 2)} + \mu P_{(n+2, 1)} + \mu P_{(n+2, 2)}\right) z^{n+2}} = \\
		= a \sum_{n=1}^{\infty}{P_{(n-1, 2)} z^{n+2}} + (1 - a)\sum_{n=1}^{\infty}{P_{(n+2, 1)} z^{n+2}} + (1 - a)\sum_{n=1}^{\infty}{P_{(n+2, 2)} z^{n+2}} = \\
		= a P_{(0, 2)} z^2 + azR(z) + (1 - a) P_{(0, 1)} \sum_{n=1}^{\infty}{a^{n+2} z^{n+2}} + (1 - a) \frac{1}{z^2}\sum_{n=1}^{\infty}{P_{(n+2, 2)} z^{n+4}} = \\ 
		= a P_{(0, 2)} z^2 + azR(z) + P_{(0, 1)} (1 - a) \frac{(az)^3}{1 - az} + (1 - a)z^{-2} R(z) - (1 - a) (P_{(1, 2)} z + P_{(2, 2)} z^2)
	\end{multline*}
	
	Отсюда
	\begin{equation*}
		(1 - az - (1-a)z^{-2})R(z) = a P_{(0, 2)} z^2 + P_{(0, 1)} (1 - a) \frac{(az)^3}{1 - az} - (1 - a) (P_{(1, 2)} z + P_{(2, 2)} z^2)
	\end{equation*}
	Выразим $R(z)$ как
	\begin{equation}\label{eq:12}
	R(z) = \frac{a P_{(0, 2)} z^2 + P_{(0, 1)} (az)^3 \frac{1 - a}{1 - az} - (1 - a) (P_{(1, 2)} z + P_{(2, 2)} z^2)}{1 - az + (a-1)z^{-2}}
	\end{equation}
	Заметим что знаменатель этого выражения обращается в нуль на корнях уравнения
	$$z^2 - az^3 + (a - 1) = 0$$
	Найти их, в зависимости от параметра $a$ нетрудно, приняв во внимание что из \eqref{eq:a}:
	$$0 < a < \frac{2}{3}$$
	И так:
	\begin{align}\label{eq:roots}
	\begin{split}
		z_0 &= 1 \\
		z_1 &= \frac{1 - a + \sqrt{(1+3a)(1-a)}}{2a} \\
		z_2 &= \frac{1 - a - \sqrt{(1+3a)(1-a)}}{2a}
	\end{split}
	\end{align}
	причем при возможных значениях $a$:
	$$(1 + 3a)(1 - a) > 0$$
	т.е. формулы корректны.
	
	Посмотрим чему равен знаменатель \eqref{eq:12} при $z = z_0 = 1$. Для простоты введем обозначения
	\begin{align}\label{eq:x}
	\begin{split}
	x_0 &= P_{(0, 0)} \\
	x_1 &= P_{(0, 1)}
	\end{split}
	\end{align}
	И так:
	\begin{equation}\label{eq:10}
		aP_{(0, 2)} + a^3 P_{(0, 1)} - (1 - a)(P_{(1, 2)} + P_{(2, 2)}) = \cdots
	\end{equation}
	Сложив 2-е и 3-е уравнения из \eqref{eq:1} получаем
	\begin{multline*}
		\mu \left(P_{(2, 2)} + P_{(1, 2)}\right) =  (\mu + \lambda)\left(P_{(0, 1)} + P_{(0, 2)}\right) - \mu P_{(1, 1)} - \lambda P_{(0, 0)} - \mu P_{(2, 1)} = \\ = (1 - a)^{-1} \lambda x_0 - \left(\mu a x_1 + \lambda x_0 + \mu a^2 x_1\right)
	\end{multline*}
	И теперь из \eqref{eq:10}:
	\begin{equation*}
		\cdots = a \left(\frac{\lambda}{\mu} x_0 - x_1\right) + a^3 x_1 - \frac{1}{\lambda + \mu} \left((1 - a)^{-1} \lambda x_0 - \left(\mu a x_1 + \lambda x_0 + \mu a^2 x_1\right)\right) = \cdots
	\end{equation*}
	Сгруппировав слагаемые получаем что
	\begin{equation*}
		\cdots = \left(a \frac{\lambda}{\mu} - \frac{\lambda}{\mu} + a\right) x_0 + \left(a^3 - a + a(1 - a) + a^2 (1 - a)\right) x_1 = 0
	\end{equation*} 
	т.е. знаменатель равен нулю и это ожидаемо в силу \eqref{eq:11} и \eqref{eq:roots}. 
	
	Легко проверить что при возможных значениях $a$:
	\begin{equation*}
		-1 < z_2 < 0
	\end{equation*}
	и с учетом \eqref{eq:9} это означает что числитель \eqref{eq:12} также должен обращаться в нуль при $z = z_2$. Упростим числитель, обозначив его за $W(z)$ т.е.
	\begin{equation*}
		R(z) = \frac{W(z)}{1 - az + (a-1)z^{-2}}
	\end{equation*} 
	Воспользуемся 3-м уравнением \eqref{eq:1}, выразив $P_{(2, 2)}$:
	\begin{equation*}
		P_{(2, 2)} = \frac{1}{\mu}\left((\mu + \lambda) P_{(0, 2)} - \mu a^2 x_1\right) = (1 - a)^{-1} \frac{\lambda}{\mu} x_0 - (1 - a)^{-1} x_1  - a^2 x_1
	\end{equation*}
	С учетом \eqref{eq:4} получаем что:
	\begin{multline*}
		W(z) = x_0 \left(a \frac{\lambda}{\mu} z^2 + a z - \frac{\lambda}{\mu}\right) + \\ + x_1 \left((az)^3 \frac{1-a}{1-az} - az^2 - z + (1-a)az + z^2 + a^2 (1 - a)z^2\right)
	\end{multline*}
	Заметим что $W(1) = 0$, это совпадает с результатом полученным ранее. 
	
	Поскольку
	$$W(z_2) = 0$$
	получаем
	\begin{equation}\label{eq:C}
		x_0 = \frac{az_2^2 + z_2 - (az_2)^3 \tfrac{1-a}{1-az_2} - (1-a)az_2 - z_2^2 - a^2 (1-a)z_2^2}{a \frac{\lambda}{\mu} z_2^2 + a z_2 - \frac{\lambda}{\mu}} x_1 = C(a) x_1
	\end{equation}
	где $C(a)$ зависит только от $a$ т.к. $z_2$ также зависит от $a$ из \eqref{eq:roots}.
	
	Теперь найдем $R(1)$. Поскольку из выражения \eqref{eq:12} оно не определено, получается неопределенность $\frac{0}{0}$, мы можем перейти к пределу и воспользоваться правилом Лопиталя:
	\begin{equation*}
		R(1) = \lim_{z \to 1}{\frac{W(z)}{1 - az + (a-1)z^{-2}}} = \lim_{z \to 1}{\frac{W^\prime(z)}{-a - 2(a-1)z^{-3}}}
	\end{equation*}
	Найдем производную числителя т.к. помимо вычисления предела, она нам еще потребуется в дальнейшем:
	\begin{multline*}
		W^\prime (z) = x_0 \left(2 a \frac{\lambda}{\mu} z + a\right) + \\ 
		+ x_1 \left(3a^3 z^2 \frac{1-a}{1-az} + (az)^3 \frac{a(1-a)}{(1-az)^2} - 2az - 1 + (1 - a)a + 2z + 2a^2 (1 - a) z \right)
	\end{multline*}
	Тогда
	\begin{equation}\label{eq:D}
		R(1) = \frac{W^\prime (1)}{2 - 3a} = x_1 \frac{C(a)\left(2a\frac{\lambda}{\mu} + a\right) + a^3 \left(1 + \frac{a}{1-a}\right) + a^2 - a +1}{2 - 3a} = x_1 D(a)
	\end{equation}
	Заметим что $2 - 3a \neq 0$ при допустимых значениях $a$.
	
	Теперь мы можем использовать условие нормировки, с учетом \eqref{eq:8}:
	\begin{equation*}
		G(1) = x_0 (1 - a)^{-1} + x_1 \frac{a}{1-a} + R(1) = \left( \frac{C(a) + a}{1-a} + D(a)\right) x_1 = 1
	\end{equation*}
	откуда
	\begin{equation}
		x_1 = \frac{1}{\frac{C(a) + a}{1-a} + D(a)}
	\end{equation}
	и соответственно
	\begin{equation}
		x_0 = \frac{C(a)}{\frac{C(a) + a}{1-a} + D(a)}
	\end{equation}
	где $C(a)$ из \eqref{eq:C}, а $D(a)$ из \eqref{eq:D}. Остальные финальные вероятности можно получить из $x_0$ и $x_1$.
	
	В заключение, найдем среднее число требований находящихся в системе в установившемся режиме:
	\begin{equation*}
		\mean{n + s} = P_{(0, 1)} \cdot 1 + P_{(0, 2)} \cdot 2 + \sum_{n=1}^{\infty}{P_{(n, 1)} (n + 1)} + \sum_{n=1}^{\infty}{P_{(n, 2)} (n+2)} = G^\prime (1)
	\end{equation*}
	или
	\begin{equation*}
		\mean{n + s} = \left(2 \frac{\lambda}{\mu} C(a) - 1\right)x_1 + \frac{a (1 - a) + a^2}{(1-a)^2} x_1 + R^\prime (1)
	\end{equation*}
	Т.к.
	\begin{equation*}
		R^\prime (z) \left(1 - az - (1-a)z^{-2}\right) + R(z) \left(-a + 2(1-a)z^{-3}\right) = W^\prime (z)
	\end{equation*}
	и 
	\begin{equation*}
		R^{\prime\prime} (z) \left(1 - az - (1-a)z^{-2}\right) + 2R^\prime (z) \left(-a + 2(1-a)z^{-3}\right) - 6(1-a)z^{-4} R(z) = W^{\prime\prime} (z)
	\end{equation*}
	нам остается найти $W^{\prime\prime} (1)$ для вычисления $R^\prime (1)$. Выражение для $W^\prime (z)$ было получено ранее, дифференцируем его:
	\begin{multline*}
		W^{\prime\prime} (z) = 2 a \frac{\lambda}{\mu} C(a) x_1 + \\
		+ x_1 \left(6a^3 \frac{1-a}{1-az} + 6a^3 z^2 \frac{(1-a)a}{(1-az)^2} + 2(az)^3 \frac{a^2 (1-a)}{(1-az)^3} - 2a + 2 + 2a^2 (1 - a)\right) 
	\end{multline*}
	Тогда
	\begin{multline*}
		W^{\prime\prime}(1) = \left(2a\frac{\lambda}{\mu} + 6a^3 + 6a^3 \frac{a}{1-a} + 2a^3 \frac{a^2}{(1-a)^2} - 2a + 2 + 2a^2 (1-a)\right)x_1 = \\
		= 2 \frac{1 - 3a + 5a^2 - 2a^3}{(1-a)^2} x_1
	\end{multline*}
	и 
	\begin{equation}\label{eq:R}
		R^\prime (1) = \frac{6 (1-a) x_1 D(a) + W^{\prime\prime} (1)}{2\left(2 - 3a\right)} = \frac{(1-a)^3 D(a) + 1 - 3a + 5a^2 - 2a^3}{(1-a)^2(2 - 3a)} x_1
	\end{equation}
	
	И так:
	\begin{equation}\label{eq:mean}
		\mean{n + s} = \frac{2 \frac{a}{1-a} C(a) - 1 + \frac{a}{(1-a)^2} + \frac{(1-a)^3 D(a) + 1 - 3a + 5a^2 - 2a^3}{(1-a)^2(2 - 3a)}}{\frac{C(a) + a}{1-a} + D(a)}
	\end{equation}
	где с учетом того что $\frac{\lambda}{\mu} = a (1-a)^{-1}$:
	\begin{align*}
		C(a) &= \frac{az_2^2 + z_2 - (az_2)^3 \tfrac{1-a}{1-az_2} - (1-a)az_2 - z_2^2 - a^2 (1-a)z_2^2}{\frac{a^2}{1-a} z_2^2 + a z_2 - \frac{a}{1-a}} \\
		D(a) &= \frac{C(a)\left(2\frac{a^2}{1-a} + a\right) + a^3 \left(1 + \frac{a}{1-a}\right) + a^2 - a +1}{2 - 3a} \\
		z_2 &= \frac{1 - a - \sqrt{(1+3a)(1-a)}}{2a}
	\end{align*}
	
	Поскольку формулы достаточно громоздкие\footnote{можно было бы и упростить, но автору не хватило терпения...}, необходимо их как-то верифицировать. Например, изучить поведение $x_0 = P_{(0, 0)}$ и $x_1 = P_{(0, 1)}$ в зависимости от $a = \frac{\lambda}{\mu + \lambda}$:
	
	\begin{figure}[h!]
		\centering
		\includegraphics[scale=0.24]{./julia_scripts/plot1.pdf}
	\end{figure}

	Видим что $P_{(0, 0)} \approx 1$ при малых значениях $a$ т.е. когда $\lambda$ много меньше $\mu$. Это ожидаемо т.к. в таком случае система практически моментально обслуживает очередное поступившее требование и большую часть времени не выходит из состояния $(0, 0)$ соответствующего пустой очереди и пустому каналу обслуживания, следовательно финальная вероятность этого состояния близка к единице. Также мы видим что как $P_{(0, 0)}$, так и $P_{(0, 1)}$ не превосходят единицы при допустимых значениях $a$. Так и должно быть, ведь это вероятности!
	
	\newpage
	
	Также построим график зависимости среднего числа требований от параметра $a = \frac{\lambda}{\mu + \lambda}$. Понятно что чем больше отношение $\frac{\lambda}{\mu}$, а пропорционально ему и $a$, тем больше требований собирается в накопителе (система не успевает их обрабатывать) и величина $\mean{n+s}$ должна отражать этот факт:
	\begin{figure}[h!]
		\centering
		\includegraphics[scale=0.24]{./julia_scripts/plot2.pdf}
	\end{figure}

	Видим что $\mean{n+s}$ монотонно возрастает и $a = \frac{2}{3}$ -- критическая точка. Действительно, в этом случае $\lambda = 2\mu$ и, как было сказано в начале решения, система становится хаотической.
\end{proof}
	
\end{document}
