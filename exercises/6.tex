\documentclass[12pt,a4paper]{article}
\usepackage[utf8]{inputenc}
\usepackage[english,russian]{babel}
\usepackage{hyperref}
\hypersetup{
	colorlinks   = true, 
	urlcolor     = blue, 
	linkcolor    = blue, 
	citecolor   = red
}
\usepackage{amsthm,amssymb,amsfonts,amsmath}
\usepackage{physics}
\usepackage{xcolor}
\usepackage[left=3cm,right=3cm,
top=3cm,bottom=3cm,bindingoffset=0cm]{geometry}
\usepackage{tcolorbox}
\usepackage{import}
\usepackage{xifthen}

\newtcolorbox{exercise}[1][]
{
	colframe = black!20,
	colback  = white,
	coltitle = black,  
	title    = {#1},
}

\newcommand{\incfig}[1]{%
	\def\svgwidth{\columnwidth}
	\import{./figures/}{#1.pdf_tex}
}
\newcommand{\pP}{\mathbf{P}}
\newcommand{\mean}[1]{\mathbf{M}[#1]}

\author{Самутичев Е.Р.}
\date{\today}
\title{}



\begin{document}
	
\maketitle

\begin{exercise}[Задача 4-4]
	Имеется техническое устройство, состоящее из $n = 9$ элементов, соединенных по следующей схеме:
	
	\incfig{2}
	
	Элементы могут отказывать по случайному закону, потоки отказов для каждого элемента -- простейшие с интенсивностью $\lambda_i$ для $i$-го элемента, элементы отказывают по одному. Требуется рассмотреть возможные состояния системы, записать уравнения Колмогорова и наметить их решение.
\end{exercise}
\begin{proof}[Решение]
	Для моделирования работы схемы выделим следующие состояния: \\
	$S_1$ -- все элементы работают \\
	$S_2$ -- не работает элемент 3 \\
	$S_3$ -- не работает элемент 4 \\
	$S_4$ -- не работает элемент 5 \\
	$S_5$ -- не работает элемент 6 \\
	$S_6$ -- не работают элементы 3, 4 \\
	$S_7$ -- не работают элементы 5, 6 \\
	$S_8$ -- схема вышла из строя
	
	Переходы между состояниями $S_i \rightarrow S_j$ можно описать таблицей:
	\begin{table}[h!]
		\centering
		\begin{tabular}{|l|l|l|}
			\hline
			$i$ & $j$ & Интенсивность \\ \hline
			1 & 2 & $\lambda_3$ \\ \hline
			1 & 3 & $\lambda_4$ \\ \hline
			1 & 4 & $\lambda_5$ \\ \hline
			1 & 5 & $\lambda_6$ \\ \hline
			1 & 8 & $\lambda_1 + \lambda_2 + \lambda_7 + \lambda_8 + \lambda_9$ \\ \hline
			2 & 6 & $\lambda_4$ \\ \hline
			2 & 8 & $\lambda_1 + \lambda_2 + \lambda_5 + \lambda_6 + \lambda_7 + \lambda_8 + \lambda_9$ \\ \hline
			3 & 6 & $\lambda_3$ \\ 
		\end{tabular}
	\end{table}
	\begin{table}[h!]
		\centering
		\begin{tabular}{|l|l|l|}
			3 & 8 & $\lambda_1 + \lambda_2 + \lambda_5 + \lambda_6 + \lambda_7 + \lambda_8 + \lambda_9$ \\ \hline
			4 & 7 & $\lambda_6$ \\ \hline
			4 & 8 & $\lambda_1 + \lambda_2 + \lambda_3 + \lambda_4 + \lambda_7 + \lambda_8 + \lambda_9$ \\ \hline
			5 & 7 & $\lambda_5$ \\ \hline
			5 & 8 & $\lambda_1 + \lambda_2 + \lambda_3 + \lambda_4 + \lambda_7 + \lambda_8 + \lambda_9$ \\ \hline 
			6 & 8 & $\lambda_1 + \lambda_2 + \lambda_5 + \lambda_6 + \lambda_7 + \lambda_8 + \lambda_9$ \\ \hline
			7 & 8 & $\lambda_1 + \lambda_2 + \lambda_3 + \lambda_4 + \lambda_7 + \lambda_8 + \lambda_9$\\ \hline
		\end{tabular}
	\end{table}
	
	Запишем уравнения Колмогорова:
	\begin{align*}
		\frac{d p_1}{d t} &= -p_1 \sum_{i=1}^{9}{\lambda_i} \\
		\frac{d p_2}{d t} &= -p_2 \sum_{\substack{i = 1 \\ i \neq 3}}^{9}{\lambda_i} + p_1 \lambda_3 \\
		\frac{d p_3}{d t} &= -p_3 \sum_{\substack{i = 1 \\ i \neq 4}}^{9}{\lambda_i} + p_1 \lambda_4 \\
		\frac{d p_4}{d t} &= -p_4 \sum_{\substack{i = 1 \\ i \neq 5}}^{9}{\lambda_i} + p_1 \lambda_5 \\
		\frac{d p_5}{d t} &= -p_5 \sum_{\substack{i = 1 \\ i \neq 6}}^{9}{\lambda_i} + p_1 \lambda_6 \\
		\frac{d p_6}{d t} &= -p_6 \sum_{\substack{i = 1 \\ i \neq 3, 4}}^{9}{\lambda_i} + p_2 \lambda_4 + p_3 \lambda_3 \\
		\frac{d p_7}{d t} &= -p_7 \sum_{\substack{i = 1 \\ i \neq 5, 6}}^{9}{\lambda_i} + p_4 \lambda_6 + p_5 \lambda_5 \\
		\frac{d p_8}{d t} &= p_1 (\lambda_1 + \lambda_2 + \lambda_7 + \lambda_8 + \lambda_9) + (p_2 + p_3 + p_6) \sum_{\substack{i = 1 \\ i \neq 3, 4}}^{9}{\lambda_i} + (p_4 + p_5 + p_7) \sum_{\substack{i = 1 \\ i \neq 5, 6}}^{9}{\lambda_i}
	\end{align*}
	с начальными условиями $p_1 (0) = 1$ и $p_i (0) = 0, i = 2, 3, ..., 8$. Последнее уравнение заменим на условие нормировки, получив
	$$p_8 (t) = 1 - \sum\limits_{i=1}^{7}{p_i (t)}$$
	
	Решение первого уравнения системы:
	$$p_1 (t) = e^{-t\sum\limits_{i=1}^{9}{\lambda_i}}$$
	Решение последующих уравнений возможно через преобразование Лапласа, в частности рассмотрим как получить решение 2-го уравнения перейдя к изображениям
	$$s P_2 (s) - p_2 (0) = -P_2 (s) \sum_{\substack{i = 1 \\ i \neq 3}}^{9}{\lambda_i} + P_1 (s) \lambda_3 = -P_2 (s) \sum_{\substack{i = 1 \\ i \neq 3}}^{9}{\lambda_i} + \frac{1}{s + \sum\limits_{i=1}^{9}{\lambda_i}}$$
	откуда
	$$P_2 (s) = \frac{1}{s + \sum\limits_{\substack{i=1 \\ i \neq 3}}^{9}{\lambda_i}} \frac{1}{s + \sum\limits_{i=1}^{9}{\lambda_i}}$$
	и вернувшись к оригиналам
	$$p_2 (t) = \int_{0}^{t}{\exp(-\tau \sum\limits_{\substack{i = 1 \\ i \neq 3}}^{9}{\lambda_i}) p_1 (t - \tau)d\tau}$$
	и т.д.
\end{proof}

\newpage

\begin{exercise}[Задача 4-5]
	Имеется техническое устройство, состоящее из $n = 9$ элементов, соединенных по следующей схеме:
	
	\incfig{2}
	
	Элементы могут отказывать по случайному закону, потоки отказов для каждого элемента -- простейшие с интенсивностью $\lambda_i$ для $i$-го элемента, элементы отказывают по одному. Схема ремонтируется, ремонт начинается после первого отказа с интенсивностью $\mu$ и продолжается до восстановления устройства. Требуется вычислить финальные вероятности возможных состояний системы.
\end{exercise}
\begin{proof}[Решение]
	Состояния выделяем как в задаче 4-4, запишем уравнения для финальных вероятностей с учетом ремонта, заменив последнее уравнение на условие нормировки:
	\begin{align*}
	&-p_1^{(\infty)} \sum_{i=1}^{9}{\lambda_i} + \mu \sum_{i=2}^{9}{p_i^{(\infty)}}= 0\\
	&-p_2^{(\infty)} \sum_{\substack{i = 1 \\ i \neq 3}}^{9}{\lambda_i} + p_1^{(\infty)} \lambda_3 - \mu p_2^{(\infty)}= 0 \\
	&-p_3^{(\infty)} \sum_{\substack{i = 1 \\ i \neq 4}}^{9}{\lambda_i} + p_1^{(\infty)} \lambda_4 - \mu p_3^{(\infty)} = 0 \\
	&-p_4^{(\infty)} \sum_{\substack{i = 1 \\ i \neq 5}}^{9}{\lambda_i} + p_1^{(\infty)} \lambda_5 - \mu p_4^{(\infty)} = 0 \\
	&-p_5^{(\infty)} \sum_{\substack{i = 1 \\ i \neq 6}}^{9}{\lambda_i} + p_1^{(\infty)} \lambda_6 - \mu p_5^{(\infty)} = 0 \\
	&-p_6^{(\infty)} \sum_{\substack{i = 1 \\ i \neq 3, 4}}^{9}{\lambda_i} + p_2^{(\infty)} \lambda_4 + p_3^{(\infty)} \lambda_3 - \mu p_6^{(\infty)} = 0 \\
	&-p_7^{(\infty)} \sum_{\substack{i = 1 \\ i \neq 5, 6}}^{9}{\lambda_i} + p_4^{(\infty)} \lambda_6 + p_5^{(\infty)} \lambda_5 - \mu p_7^{(\infty)} = 0 \\
	&\sum_{i=1}^{8}{p_i^{(\infty)}} = 1
	\end{align*}
	Подставляя условие нормировки в первое уравнение, получаем:
	\begin{equation*}
		-p_1^{(\infty)} \sum_{i=1}^{9}{\lambda_i} + \mu (1 - p_1^{(\infty)}) = 0
	\end{equation*}
	Т.е.
	\begin{equation*}
		p_1^{(\infty)} = \frac{\mu}{\sum_{i=1}^{9}{\lambda_i} + \mu}
	\end{equation*}
	Остальные финальные вероятности последовательно выражаются, предыдущие через последующие:
	\begin{align*}
		p_2^{(\infty)} = \frac{\lambda_3 p_1^{(\infty)}}{\sum_{\substack{i = 1 \\ i \neq 3}}^{9}{\lambda_i} + \mu} \\
		p_3^{(\infty)} = \frac{\lambda_4 p_1^{(\infty)}}{\sum_{\substack{i = 1 \\ i \neq 4}}^{9}{\lambda_i} + \mu} \\
		p_4^{(\infty)} = \frac{\lambda_5 p_1^{(\infty)}}{\sum_{\substack{i = 1 \\ i \neq 5}}^{9}{\lambda_i} + \mu} \\
		p_5^{(\infty)} = \frac{\lambda_6 p_1^{(\infty)}}{\sum_{\substack{i = 1 \\ i \neq 6}}^{9}{\lambda_i} + \mu} \\
		p_6^{(\infty)} = \frac{\lambda_4 p_2^{(\infty)} + \lambda_3 p_3^{(\infty)}}{\sum_{\substack{i = 1 \\ i \neq 3, 4}}^{9}{\lambda_i} + \mu} \\
		p_7^{(\infty)} = \frac{\lambda_6 p_4^{(\infty)} + \lambda_5 p_5^{(\infty)}}{\sum_{\substack{i = 1 \\ i \neq 5, 6}}^{9}{\lambda_i} + \mu} \\
		p_8^{(\infty)} = 1 - \sum_{i=1}^{7}{p_i^{(\infty)}}
	\end{align*}
\end{proof}

\end{document}
