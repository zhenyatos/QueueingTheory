\documentclass[12pt,a4paper]{article}
\usepackage[utf8]{inputenc}
\usepackage[english,russian]{babel}
\usepackage{hyperref}
\hypersetup{
	colorlinks   = true, 
	urlcolor     = blue, 
	linkcolor    = blue, 
	citecolor   = red
}
\usepackage{amsthm,amssymb,amsfonts,amsmath}
\usepackage{physics}
\usepackage{xcolor}
\usepackage[left=3cm,right=3cm,
top=3cm,bottom=3cm,bindingoffset=0cm]{geometry}
\usepackage{tcolorbox}
\usepackage{import}
\usepackage{xifthen}

\newtcolorbox{exercise}[1][]
{
	colframe = black!20,
	colback  = white,
	coltitle = black,  
	title    = {#1},
}

\newcommand{\incfig}[1]{%
	\def\svgwidth{\columnwidth}
	\import{./figures/}{#1.pdf_tex}
}
\newcommand{\pP}{\mathbf{P}}

\author{Самутичев Е.Р.}
\date{\today}
\title{}



\begin{document}
	
\maketitle

\begin{exercise}[Задача 5]
	Имеется поток Эрланга второго порядка ($r = 2$) интенсивности $\lambda$, для которого известно что при наблюдении за ним в течение времени $t_0$ вероятность не получить ни одного требования и вероятность получить хотя бы одно требование совпадают. Какова для этого потока вероятность не получить ни одного требования за время $2t_0$?
\end{exercise}
\begin{proof}[Решение]
	В силу стационарности потока Эрланга, можно считать что наблюдение осуществляется во временном промежутке $[0, t_0]$, т.е. нас интересует интервал $\tau_1$ до появления первого события потока от начала отсчета времени. По условию $p_0 (t_0) = \pP\{\tau_1 > t_0\} = \pP\{\tau_1 \leq t_0\} = p_1 (t_0)$, что равносильно
	$$1 - F(t_0) = F(t_0)$$
	где $F(x)$ обозначает функцию распределения Эрланга, в данном случае
	$$F(x) = 1 - e^{-2\lambda x} (1 + 2\lambda x)$$
	т.е. $t_0 > 0$ удовлетворяет соотношению
	\begin{equation}\label{eq:1}
		2e^{-2\lambda t_0} (1 + 2\lambda t_0) = 1
	\end{equation}
	Обозначим $2\lambda t_0 = x_*$, тогда \eqref{eq:1} равносильно
	\begin{equation}\label{eq:2}
		2 + 2x_* = e^{x_*}
	\end{equation}
	Введем функцию $\varphi(x) = 2 + 2x - e^x$ и заметим что она непрерывна, притом т.к. $2 < e < 3$:
	\begin{align*}
		\varphi(1) &= 4 - e > 0 \\
		\varphi(3) &= 8 - e^3 < 0
	\end{align*}
	и по теореме о промежуточном значении получаем что $\exists x_* \in [1, 3]: \varphi(x_*) = 0$, а значит решение уравнения \eqref{eq:2} существует в области $x_* > 0$. Более того можно показать что оно единственно. Тогда искомая вероятность
	$$p_0 (2t_0) = \pP\{\tau_1 > 2t_0\} = 1 - F(2t_0) = e^{-2x_*} (1 + 2x_*)$$
\end{proof}

\begin{exercise}[Задача 4-2]
	Имеется техническое устройство, состоящее из $n = 9$ элементов, соединенных по следующей схеме:
	
	\incfig{2}
	
	Элементы могут отказывать по случайному закону, потоки отказов для каждого элемента -- потоки Эрланга второго порядка с интенсивностью $\lambda$. Определить вероятность безотказной работы всей схемы в целом в интервале $[0, t]$ и среднее время безотказной работы. Элементы, включенные параллельно, дублируют друг друга, и для работы схемы достаточно, чтобы работал хотя бы один элемент из параллельно включенных.
\end{exercise}
\begin{proof}[Решение]
	Введем события $A_i$ -- безотказная работа $i$-го элемента в $[0, t]$; $A$ -- безотказная работа всей системы в $[0, t]$. Поскольку потоки отказов являются потоками Эрланга, то 
	$$\pP(A_i) = p_0 (0, t) = \pP\{\tau_1 > t\} = 1 - F_\text{э}(t) = e^{-2 \lambda t} (1 + 2\lambda t) = \varphi(t)$$
	где $F_\text{э}$ -- функция распределения Эрланга с интенсивностью $\lambda$ и порядком $r$. Аналогично задаче 4-1 получаем
	$$\pP(A) = \varphi^5 (t) (2 \varphi^2 (t) - \varphi^4 (t)) = 2 \varphi^7 (t) - \varphi^9 (t)$$
	и тогда среднее время безотказной работы:
	$$\bar{t} = \int_{0}^{\infty}{(2\varphi^7 (t) - \varphi^9 (t)) dt}$$
	Введем вспомогательные интегралы
	$$I_k = \int_{0}^{\infty}{\varphi^k(t) dt}$$
	через которые
	$$\bar{t} = 2I_7 - I_9$$
	Можно показать что
	$$I_k = \frac{1}{2k}\sum_{j=0}^{k}{\frac{k!}{k^j (k - j)!}} \bar{\tau}$$
	где $\bar{\tau} = \frac{1}{\lambda}$ -- среднее значение безотказной работы одного элемента. Вычислим\footnote{был использован язык \textbf{Julia} для вычислительной математики} необходимые нам значения:
	$$I_7 = \frac{236365}{823543}\bar{\tau} \approx 0.28701\bar{\tau}$$
	$$I_9 = \frac{10661993}{43046721}\bar{\tau} \approx 0.24768\bar{\tau}$$
	тогда
	$$\bar{t} \approx 0.32634\bar{\tau}$$
	Таким образом при переходе от простейшего потока ($r = 1$) к эрланговскому второго порядка -- отношение среднего времени работы системы к среднему времени работы одного элемента увеличилось в $\approx 1.86$ раз.
\end{proof}

\end{document}
