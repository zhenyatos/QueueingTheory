\documentclass[12pt,a4paper]{article}
\usepackage[utf8]{inputenc}
\usepackage[english,russian]{babel}
\usepackage{hyperref}
\hypersetup{
	colorlinks   = true, 
	urlcolor     = blue, 
	linkcolor    = blue, 
	citecolor   = red
}
\usepackage{amsthm,amssymb,amsfonts,amsmath}
\usepackage{physics}
\usepackage{xcolor}
\usepackage[left=3cm,right=3cm,
top=3cm,bottom=3cm,bindingoffset=0cm]{geometry}
\usepackage{tcolorbox}
\usepackage{import}
\usepackage{xifthen}

\newtcolorbox{exercise}[1][]
{
	colframe = black!20,
	colback  = white,
	coltitle = black,  
	title    = {#1},
}

\newcommand{\incfig}[1]{%
	\def\svgwidth{\columnwidth}
	\import{./figures/}{#1.pdf_tex}
}
\newcommand{\pP}{\mathbf{P}}

\author{Самутичев Е.Р.}
\date{\today}
\title{}



\begin{document}
	
\maketitle

\begin{exercise}[Задача 3]
	Рассмотрим простейший поток событий с интенсивностью $\lambda$. Пусть $N(t)$ обозначает число событий, произошедших к моменту $t$, а отношение 
	$$\tilde{\lambda}(t) = \frac{N(t)}{t}$$
	-- среднюю интенсивность потока по результатам наблюдения за ним в интервале $[0, t]$. Доказать, что $\tilde{\lambda} \underset{t\to\infty}{\to} \lambda$ в смысле среднего квадратического, т.е.
	$$\lim_{t\to\infty}{\textbf{M}[(\tilde{\lambda} - \lambda)^2]} = 0$$
\end{exercise}
\begin{proof}[Решение]
	Поскольку простейший поток является пуассоновским, то
	$$p_k (0, t) = e^{-\lambda t} \frac{(\lambda t)^k}{k!}$$ 
	Значит случайная величина $N(t)$ распределена по Пуассону с параметром $\lambda t$. Математическое ожидание и дисперсия такой случайной величины нам известны, получаем
	$$\bar{n}(t) = \textbf{M}[N(t)] = \lambda t$$
	и
	$$\textbf{M}[(N(t) - \bar{n}(t))^2] = \textbf{D}[N(t)] = \lambda t$$
	Наконец
	$$\textbf{M}[(\tilde{\lambda} - \lambda)^2] = \textbf{M}\left[\left(\frac{N(t)}{t} - \frac{\bar{n}(t)}{t}\right)^2\right] = \frac{1}{t^2} \textbf{M}[(N(t) - \bar{n}(t))^2] = \frac{\lambda}{t}$$
	и совершив предельный переход
	$$\lim_{t \to \infty}{\textbf{M}[(\tilde{\lambda} - \lambda)^2]} = \lim_{t \to \infty}{\frac{\lambda}{t}} = 0$$
	что и требовалось доказать.
\end{proof}

\begin{exercise}[Задача 4]
	Имеется техническое устройство, состоящее из $n=9$ элементов, соединенных по следующей схеме: \\
	
	\incfig{2}
	
	Элементы могут отказывать по случайному закону, потоки отказов для каждого элемента -- простейшие с интенсивностью $\lambda$. Определить вероятность безотказной работы всей схемы в целом в интервале $[0, t]$ и среднее время безотказной работы. Элементы, включенные параллельно, дублируют друг друга, и для работы схемы достаточно, чтобы работал хотя бы один элемент из параллельно включенных.
\end{exercise}
\begin{proof}[Решение]
	Введем события: $A_i$ -- безотказная работа $i$-го элемента в $[0, t]$; $A$ -- безотказная работа всей системы в $[0, t]$. Согласно условию, введенные события связаны равенством
	$$A = A_1 A_2 (A_3 A_4 + A_5 A_6) A_7 A_8 A_9$$
	Поскольку потоки отказов простейшие, то
	$$\pP(A_i) = p_0 (0, t) = e^{-\lambda t}$$
	Отказы различных элементов системы \underline{независимы} т.е. события $A_i$ независимы, тогда
	$$\pP(A) = \pP(A_1) \pP(A_2) \pP(A_3 A_4 + A_5 A_6) \pP(A_7) \pP(A_8) \pP(A_9)$$
	т.к. 
	$$\pP(A_3 A_4 + A_5 A_6) = \pP(A_3 A_4) + \pP(A_5 A_6) - \pP(A_3 A_4 A_5 A_6) = 2e^{-2\lambda t} - e^{-4\lambda t}$$
	и наконец получаем
	$$\pP(A) = e^{-5\lambda t}\left(2e^{-2\lambda t} - e^{-4\lambda t}\right) = 2e^{-7\lambda t} - e^{-9\lambda t}$$
	Пусть $T$ -- случайная величина, время безотказной работы схемы. Будем искать среднее время $\bar{t}$ безотказной работы. Обозначим через $F(t)$ -- функцию распределения $T$. По определению функции распределения и тому как мы задали $A$, имеем:
	$$\pP(A) = \pP\{T > t\} = 1 - F(t)$$
	$f(t) = \frac{dF (t)}{dt}$ -- плотность вероятности $T$, по определению:
	$$\bar{t} = \mathbf{M}(T) = \int_{0}^{\infty}{t f(t) dt}$$
	Представим данный интеграл в виде
	$$\bar{t} = \int_{0}^{\infty}{t d(F(t) - 1)}$$
	и воспользуемся интегрированием по частям:
	\begin{equation*}
		\bar{t} = t(F(t) - 1) \rvert_{0}^{\infty} - \int_{0}^{\infty}{(F(t) - 1)dt} = \cdots
	\end{equation*}
	т.к. $\lim\limits_{t \to +\infty}{F(t)} = 1$
	\begin{align*}
		\cdots &= 0 + \int_{0}^{\infty}{(1 - F(t))dt} = \\
				&= \int_{0}^{\infty}{\pP(A)dt} = \int_{0}^{\infty}{\left(2e^{-7\lambda t} - e^{-9\lambda t}\right) dt} = \\
				&= \frac{2}{7 \lambda} - \frac{1}{9 \lambda} = \frac{11}{63 \lambda} = \frac{11}{63}\bar{\tau} \approx 0.175\bar{\tau}
	\end{align*}
	где $\bar{\tau}$ -- среднее время работы одного элемента, а для простейшего потока \\ $\bar{\tau} = \frac{1}{\lambda}$
\end{proof}
\end{document}
