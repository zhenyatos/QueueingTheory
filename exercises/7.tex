\documentclass[12pt,a4paper]{article}
\usepackage[utf8]{inputenc}
\usepackage[english,russian]{babel}
\usepackage{hyperref}
\hypersetup{
	colorlinks   = true, 
	urlcolor     = blue, 
	linkcolor    = blue, 
	citecolor   = red
}
\usepackage{amsthm,amssymb,amsfonts,amsmath}
\usepackage{physics}
\usepackage{xcolor}
\usepackage[left=3cm,right=3cm,
top=3cm,bottom=3cm,bindingoffset=0cm]{geometry}
\usepackage{tcolorbox}
\usepackage{import}
\usepackage{xifthen}

\newtcolorbox{exercise}[1][]
{
	colframe = black!20,
	colback  = white,
	coltitle = black,  
	title    = {#1},
}

\newcommand{\incfig}[1]{%
	\def\svgwidth{\columnwidth}
	\import{./figures/}{#1.pdf_tex}
}
\newcommand{\pP}{\mathbf{P}}
\newcommand{\mean}[1]{\mathbf{M}[#1]}

\author{Самутичев Е.Р.}
\date{\today}
\title{}



\begin{document}
	
\maketitle

\begin{exercise}[Задача 7]
	Имеется процесс гибели-размножения характеризуемый следующими интенсивностями
	$$\lambda_k = \begin{cases}
		\alpha, &0 \leq k \leq m-1 \\
		0, &k \geq m
	\end{cases}$$
	$$\mu_k = \begin{cases}
		0, &k > m \\
		\beta, &1 \leq k \leq m
	\end{cases}$$
	Определить финальные вероятности этого процесса $\{p_k\}_{k=0}^{m}$, а также средний установившийся размер популяции.
\end{exercise}
\begin{proof}[Решение]
	Изобразим размеченный граф состояний, соответствующий данному процессу
	\begin{figure}[h!]
		\centering
		\incfig{3}
	\end{figure}

	Воспользуемся известной формулой для финальных вероятностей процесса гибели-размножения:
	$$p_n = p_0 \sum_{j=0}^{n}{\frac{\mu_{j+1}}{\lambda_j}} = p_0 \left(\frac{\beta}{\alpha}\right)^n = p_0 \rho^k$$
	где $\rho = \frac{\beta}{\alpha}$ -- коэффициент использования системы. Запишем условие нормировки
	$$\sum_{k=0}^{m}{p_0 \rho^k} = 1$$
	т.е. 
	$$p_0 \frac{1 - \rho^{m+1}}{1 - \rho} = 1$$
	откуда
	$$p_0 = \frac{1 - \rho}{1 - \rho^{m+1}}$$
	
	Найдем средний установившийся размер популяции:
	\begin{multline*}
		\bar{n} = \mean{N} = \sum_{k=0}^{m}{k p_k} = p_0 \sum_{k=1}^{m}{k \rho^k} = p_0 \rho \frac{1 - (m+1)\rho^m + m\rho^{m+1}}{(1-\rho)^2} = \\ = \rho \frac{p_0 + m p_0 \rho^{m+1} - (m+1)p_m}{(1-\rho)^2} = \rho \frac{1 - \rho}{1 - \rho^{m+1}} \frac{1 - (m+1)\rho^m + m\rho^{m+1}}{(1-\rho)^2} = \\ = \rho \frac{1 - (m+1)\rho^m + m\rho^{m+1}}{(1-\rho)(1 - \rho^{m+1})} = \frac{\beta}{\alpha - \beta} \frac{1 - (m+1)\rho^m + m\rho^{m+1}}{1 - \rho^{m+1}} = \\ = \frac{\beta}{\alpha - \beta} \frac{1 - (m+1)\rho^m + (m+1)\rho^{m+1} - \rho^{m+1}}{1 - \rho^{m+1}} = \\ = \frac{\beta}{\alpha - \beta} \left[1 - (m+1) \rho^m \frac{1 - \rho}{1-\rho^{m+1}}\right] = \frac{\alpha}{\alpha - \beta}\left[1 - (m+1)p_m\right]
	\end{multline*}
\end{proof}
	
\end{document}
