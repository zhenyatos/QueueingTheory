\mysection{Лекция 2}

\subsection{Входящий поток требований}

Входящий поток требований с математической точки зрения представляет разновидность \underline{потока событий} (или \underline{случайного потока}).

\begin{definition}
	{\color{red}Потоком событий} называют последовательность однородных событий, происходящих друг за другом в случайные моменты времени.
\end{definition}

\begin{figure}[h!]
	\scalebox{0.8}{\incfig{2}}
	\centering
	\caption{К определению потока событий}
\end{figure}

Геометрически поток событий можно представить как некоторое счетное множество точек числовой оси. Образующие поток события называются \\ \underline{элементарными событиями потока}, моменты их появления $T_k$ представляют собой некоторые \underline{случайные величины}.

Формально, задать поток можно двумя способами. \\
\textbf{1.} Обозначим через $X(t_1, t_2)$ число событий, произошедших в интервале \\ $[t_1, t_1 + t_2]$. Тогда задать поток -- это значит задать все вероятности
$$p_k (t_1, t_2) = \mathcal{P}\{X(t_1, t_2) = k\}, k = 0, 1, 2, ...$$
Полученная модель получается весьма громоздкой т.к. практическое нахождение функций $p_k (t_1, t_2)$ из эксперимента вызывает затруднения. Упрощение модели возможно если мы имеем дело с потоками особого вида, например со \underline{стационарными} потоками -- данный термин будет раскрыт далее.
\textbf{2.} Другой способ описания потока состоит во введении интервалов между событиями
$$\tau_k = T_k - T_{k-1}, k = 1, 2, ...$$
и задании совместного закона распределения системы случайных величин $\{\tau_k\}_{k=1}^{\infty}$

Теперь мы можем перейти к различным свойствам, которыми могут обладать потоки событий. Начнем с определения стационарности.

\begin{definition}
	Поток называется {\color{red}стационарным}, если величины $X(t_1, t_2)$ с одинаковым значением ширины $t_2$ отрезка времени $[t_1, t_1 + t_2]$, но разным значением начала отсчета времени $t_1$, будут иметь один и тот же закон распределения. Иначе говоря
	$$p_k (t_1, t_2) = p_k (t_2), k = 0, 1, 2, ...$$
\end{definition}

Другим важным свойством случайных потоков является отсутствие в них \underline{последействия} (т.е. памяти). 

\begin{definition}
	Рассмотрим два промежутка времени $\Omega_1 = [t_1, t_1 + t_2]$ и $\Omega_2 = [t_3, t_3 + t_4]$ и предположим что $\Omega_1 \cap \Omega_2 = \varnothing$. Определим случайные величины
	$X_1 = X(t_1, t_2)$ и $X_2 = X(t_3, t_4)$. Говорят что мы имеем дело с потоком {\color{red}без последействия}, если случайные величины $X_1$ и $X_2$ независимы для любых непересекающихся интервалов $\Omega_1, \Omega_2$
\end{definition}

Здесь необходимо напомнить одно важное понятие из теории вероятностей.

\begin{definition}
	{\color{red}Случайным процессом} называют функцию от момента времени $t$, значениями $U(t)$ которой являются случайные величины.
\end{definition}

Нас будут интересовать случайные процессы особого вида, т.н. марковские процессы.

\begin{definition}
	Пусть $t$ -- настоящий момент времени, $\tau$ -- некоторый будущий момент, а $s$ -- прошедший, т.е. 
	$$s < t < \tau$$
	Случайный процесс называется {\color{red}марковским процессом} если задание $t$ и $U(t)$ однозначно определяет закон распределения $U(\tau)$, соответственно $U(\tau)$ не зависит от $U(s)$. Иначе говоря будущее состояние марковского процесса полностью определяется его состоянием в настоящий момент и не зависит от прошлого.
\end{definition}

Вернемся к рассмотрению случайных потоков. Введем случайный процесс $N(t)$, определяемый как общее число требований, полученных к моменту $t$ т.е. 
$$N(t) = X(0, t)$$
Для потоков без последействия -- данный процесс будет марковским т.к. в этом случае закон распределения $N(\tau)$ может зависеть лишь от того наступило ли событие в момент $t$, а для любых $t_1, t_2$ таких что $[t_1, t_1 + t_2] \subset (t, \infty)$ случайная величина $X(t_1, t_2)$ не зависит от $N(0, t)$ по определению потока без последействия. Иначе говоря количество событий после наступления момента времени $t$ не зависит от количества событий к моменту $t$.

\subsection{Ординарные потоки требований. Интенсивность потока}

Введем вероятность попадания на интервал $[t_1, t_1 + t_2]$ не менее двух требований
$$p_{>1}(t_1, t_2) = \mathcal{P}\{X(t_1, t_2) > 1\} = \sum_{k=2}^{\infty}{p_k (t_1, t_2)}$$

\begin{definition}
	Поток событий называется {\color{red}ординарным}, если для каждого интервала $[t, t + dt]$ выполнены следующие условия
	\begin{align*}
		p_1 (t, dt) &= \lambda(t) dt + o(dt), dt \to 0 \\
		p_{>1}(t, dt) &= o(dt), dt \to 0
	\end{align*}
	где $\lambda(t)$ -- некоторая интегрируемая функция
\end{definition}

Выясним физический смысл $\lambda(t)$ как коэффициента при $dt$, введем математическое ожидание числа требований $X(t_1, t_2)$ попавших на интервал $[t_1, t_1 + t_2]$:
$$\bar{x}(t_1, t_2) = M[X(t_1, t_2)] = \sum_{k=0}^{\infty}{k p_k (t_1, t_2)}$$
Допустим что оно существует для рассматриваемого потока, т.е. ряд сходится и перепишем сумму в виде
\begin{equation}\label{eq:1}
	\bar{x}(t_1, t_2) = p_1 (t_1, t_2) + \bar{x}_0 p_{>1} (t_1, t_2)
\end{equation}

где $\bar{x}_0 (t_1, t_2) = \sum_{k=2}^{\infty}{k \frac{p_k (t_1, t_2)}{p_{>1} (t_1, t_2)}}$
Рассмотрим величины $q_k (t_1, t_2) = \frac{p_k (t_1, t_2)}{p_{>1} (t_1, t_2)}, k \geq 2$
Ясно что они неотрицательны и удовлетворяют условию нормировки
$$\sum_{k=2}^{\infty}{q_k (t_1, t_2)} = 1$$
Следовательно $q_k$ можно трактовать как некоторые вероятности, а именно как условные
$$q_k = \mathcal{P}\left(X(t_1, t_2) = k | X(t_1, t_2) \geq 2\right)$$
Это означает что $\bar{x}_0$ равняется среднему числу полученных требований при условии что их не меньше двух. Запишем \eqref{eq:1} для интервала $[t, t + dt]$, имеем 
$$\bar{x}(t, dt) = p_1 (t, dt) + \bar{x}_0 (t, dt)p_{>1}(t, dt)$$
Для большинства реальных потоков величина $\bar{x}_0 (t, dt)$ остается ограниченной при $dt \to 0$ и тогда в предположении ординарности потока получаем
\begin{equation}\label{eq:2}
	\bar{x}(t, dt) = \lambda(t)dt + o(dt) + \bar{x}_0 (t, dt) o(t) = \lambda(t)dt + o(dt)
\end{equation}
Вновь обратимся к общему числу требованй $N(t)$ полученных к моменту $t$. Это случайная величина, а среднее число таких требований
$$\bar{n}(t) = M[N(t)] = \bar{x}(0, t)$$
Нетрудно понять\footnote{довольно нестрого, здесь надо бы раскрыть что $M[N(t + dt) - N(t)] = M[X(t, dt)]$ при $dt \to 0$ т.е. есть некая сходимость}, что 
$$d\bar{n}(t) = \bar{n}(t + dt) - \bar{n}(t) = \bar{x}(t, dt)$$
Следовательно на основании \eqref{eq:2} получим
$$d\bar{n}(t) = \lambda(t)dt + o(dt)$$
Наконец, поделив обе части на $dt$ и перейдя к пределу $dt \to 0$, получаем дифференциальное уравнение
\begin{equation}
	\frac{d\bar{n}(t)}{dt} = \lambda(t)
\end{equation}
\begin{definition}\label{eq:3}
	Среднее число событий, происходящих в случайном потоке в единицу времени, называется {\color{red}интенсивностью} этого потока.
\end{definition}
И в сущности мы показали что $\lambda(t)$ в определении ординарного потока -- это интенсивность.

Допустим, что наблюдение за потоком началось при $t = 0$, причем в этот момент не было получено ни одного требования, т.е. $N(0) = 0$, тогда среднее число требований
$\bar{n}(0) = 0$ и решая дифференциальное уравнение \eqref{eq:3} получаем
$$\bar{n}(t) = \int_{0}^{t}{\lambda(s)ds} = \Lambda(t)$$
где функция $\Lambda(t) = \int\limits_{0}^{t}{\lambda(s)ds}$ называется \underline{ведущей функцией потока}. Она представляет собой среднее число требований, попавших на интервал $[0, t]$