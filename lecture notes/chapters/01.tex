\mysection{Лекция 1}

\subsection{Понятие обслуживания и системы обслуживания}

\begin{definition}
	{\color{red}Обслуживанием} называют многократное выполнение однотипных разовых работ по заявкам (запросам), поступающим в произвольные моменты времени. При этом длительность каждой из работ также может быть произвольной.
\end{definition}

Раскроем содержание определения:
\begin{itemize}
	\item Речь идет именно о \underline{многократном} поступлении заявок на обслуживание, то есть о целом \underline{потоке} заявок. Если заявки поступают редко и в небольшом количестве, то каких-либо серьезных математических проблем при анализе таких систем не возникает;
	\item Заявки могут поступать именно в \underline{произвольные} моменты времени и каждая заявка выполняется также в течение заранее неизвестного времени. В результате возможны такие эффекты, как возникновение очереди заявок на обслуживание, простои системы и т.п.
	\item Абстрагируемся от физического содержания процесса. В абстрактных математических моделях процесса обслуживания основной его характеристикой является \underline{затраченное время}. Обслужить данный объект -- это значит определенным образом его видеоизменить и переработать в соответствии с заявкой, причем контролируя прпи этом затраченное время.
\end{itemize}

В силу наличия бв процессе обслуживания множества случайных факторов, мы приходим к необходимости использования в ТМО \underline{вероятностных моделей}.

\begin{definition}
	Совокупность сил и средств (включая и человека), организационно объединенная в единое целое и предназначенная для целей обслуживания, называется {\color{red}системой обслуживания}.
\end{definition}

Системы обслуживания являются основным предметом изучения в ТМО. С учетом сказанного выше о случайных факторах при обслуживании, ТМО представляет собой некоторый специфический раздел \underline{теории вероятностей}. При применении методов ТМО реальные системы обслуживания заменяются некоторыми математическими моделями. Здесь необходимо ввести еще одно важное понятие.

\begin{definition}
	Системой массового обслуживания (или сокращенно СМО) называют составленную по особым правилам математическую модель реальной системы обслуживания.
\end{definition}

В настоящий момент разработан и детально изучен широкий класс самых разнообразных СМО. Исходя из потребностей практики, постоянно появляются новые модели СМО, продолжается изучение ранее предложенных моделей. Всё это позволяет решать большое число различных прикладных задач, связанных с обслуживанием.
\newpage
Рассмотрим типовую схему простейшей СМО:
\begin{figure}[h!]
	\incfig{1}
	\centering
	\caption{Общая схема СМО: $k$ -- емкость системы, $m$ -- число каналов обслуживания, $\ell$ -- число источников нагрузки, $k_0 = k - m$ -- емкость накопителя}
	\label{img:1}
\end{figure}

\underline{Входящий поток} -- случайный процесс, моделирующий поступление требований в систему. Является суммой элементарных потоков заявок от всех потенциальных клиентов, которых называют \underline{источниками} нагрузки. Во многих задачах их число $\ell$ настолько велико, что рассматривают асимптотический случай $\ell = \infty$, хотя есть и задачи где принципиально важно учитывать что $\ell$ ограничено (задачи об обслуживании станков), но уже при $\ell$ порядка 40-50 результаты совпадают с $\ell = \infty$.

В любой СМО обязательно присутствует \underline{рабочая часть} системы где непосредственно выполняются заявки на обслуживание.

\begin{definition}
	Совокупность элементов рабочей части, необходимых для полного обслуживания одного требования называется {\color{red}каналом обслуживания}.
\end{definition}

Канал обслуживания включает всё технологическое оборудование и всех работников, которые привлекаются для выполнения отдельно взятой разовой работы.

\begin{definition}
	{\color{red}Числом каналов обслуживания} $m$ называют максимально возможное число одновременно обслуживаемых требований.
\end{definition}

\begin{itemize}
	\item $m = 1$ -- одноканальная система;
	\item $1 < m < \infty$ -- многоканальная система;
	\item $m = \infty$ -- система немедленного обслуживания.
\end{itemize}

Система немедленного обслуживания может быть реализована на практике, если постоянно держать в резерве достаточное число временных работников, готовых в любой момент включиться в работу, а также соответствующее оборудование.

Помимо входящего потока и рабочей вчасти, во многих системах присутствует еще один, третий основной элемент -- \underline{накопитель очереди}. Его функция -- сохранять требования, ожидающие обслуживания, и передавать их в надлежащем порядке в рабочую часть для обработки.

\begin{definition}
	Максимально возможное число требований $k$, которые одновременно могут находиться в системе, называется {\color{red}емкостью системы}.
\end{definition}

$k_0 = k - m$ т.к. при максимальном заполнении системы будут заняты и все каналы обслуживания. На практике значение $k_0$ всегда ограничено, но в общей теории рассматривают и предельный случай $k_0 = \infty$ (когда вероятность переполнения накопителя крайне мала).

\subsection{Характеристики состояния СМО}

Работа системы, представленной на \hyperref[img:1]{рис. 1}, осуществляется следующим образом: если вновь пришедшее требование застает хотя бы один свободный канал обслуживания, то оно сразу без ожидания поступает на обслуживание, то оно сразу без ожидания поступает на обслуживание. Если же все каналы заняты, но есть свободное место в накопителе, то требование встает в очередь. Наконец, если заняты все места ожидания и все каналы, то требование отправляется в \underline{поток потерь}. Полностью обслуженное требование попадает в \underline{выходящий поток}.

Состояние СМО характеризуется следующими величинами:
\begin{itemize}
	\item $N(t)$ -- число требований в системе;
	\item $N_{\text{зк}} (t)$ -- число занятых каналов обслуживания;
	\item $N_{\text{оч}} (t)$ -- длина очереди в накопителе;
	\item $N_{\text{вx}} (t)$ -- число требований, поступивших на вход;
	\item $N_{\text{вып}} (t)$ -- число выполненных требований;
	\item $N_{\text{пот}} (t)$ -- число потерянных требований;
	\item $N_{\text{нак}} (t)$ -- число требований, прошедших через накопитель.
\end{itemize}

Одной из основных является проблема расчета \underline{длины очереди}.

\subsection{Типы систем обслуживания}

\begin{enumerate}
	\item $k = m, k_0 = 0$ -- отсутствие накопителя, если все каналы обслуживания заняты, то пришедшее требование будет утеряно, это \underline{система с потерями};
	\item $k_0 = \infty$ -- неограниченный накопитель, все вновь пришедшие требования могут встать в очередь и дожидаться обслуживания, это \underline{система с ожиданием};
	\item $0 < k_0 < \infty$ -- \underline{система с ожиданием и потерями}, промежуточный вариант;
	\item $m = \infty$ -- нет ни ожидания, ни потерь, а все требования, поступившие на вход системы, сразу направляются в один из каналов обслуживания, это \underline{система немедленного обслуживания}. 
\end{enumerate}